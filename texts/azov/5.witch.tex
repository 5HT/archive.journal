\section{Я Собака Азову}

Я собака Азову на прізвисько Відьма,\\
Мій господар вкривавлений поруч лежить,\\
Ми їх знищили всіх, ми вдарили з півдня,\\
Де вони не чакали нас в прокляту мить.
\\
\\
Маріуполь горів, та крізь гарь я почула їх запах,\\
Через безвіст завалів і снів чорноту,\\
Їхня кров, що навік запеклася на лапах,\\
Завжди чітко виводила нашу чету.
\\
\\
Наші кулі завжди були швидші за їхні,\\
І гостріші і непомітніші наші ножі,\\
І прокляття для проклятих — прізвисько Відьма,\\
Мала власну методу своїх ворожінь
\\
\\
Я їх чула по запаху страху і гнилі,\\
Ще з тих пір як лютневий вкривавило сніг,\\
Проти помсти відьомської кулі безсилі,\\
Варто було лиш схотіти мені.
\\
\\
І я дуже хотіла, щоб прокляті мерли,\\
В булькотінні кривавім без каяття,\\
Аби їх матері, що зродилися стервом,\\
Не знаходили більше для себе життя.
\\
\\
Я хотіла — так сталось — всі мертві навколо,\\
Я дивлюся у їхні очі порожні й дурні,\\
Наїдайтесь неситі прокляттям Азову,\\
Чом не раді? Дивіться як радо мені.
\\
\\
Рада відьма, бо відає, знає,\\
Як горлянки проклятим наповнити вщент,\\
І героям показує сходи до раю,\\
Ворогам Відьма завжди показує смерть.
\\
\\
Ще малим цуценям, коли маму смоктала,\\
Я всмоктала: людина — господар і пан,\\
І коли мене діти в дворі ображали,\\
Мій азовець мене не лишив сам на сам.
\\
\\
Ми міцніли разом синьожовтим вростаючи в землю\\
Краще котрої будь-який всесвіт не знав,\\
Поруч тебе я вила ой верше до темна,\\
Потім вишкіл, плече побратима, війна.
\\
\\
І людина моя проти нелюду гордо повстала,\\
І з моєю людиною Відьма поруч була,\\
Його руни на тілі мені крізь усі перепони сіяли,\\
І ми всесвіт разом рятували від зла.
\\
\\
Не біда, що ти впав за сорочку, господар,\\
Я тягтиму тебе в катакомби-льохи,\\
"Ми не вмремо, бо ми є солдати Азову"\\
— Саме так ти казав мені в миті лихі.
\\
\\
За сорочку зубами... бо то хіба вперше,\\
Рятувати з тобов одне одного нам,\\
Заспіваємо ще ми "Ой верше мій, верше..."\\
І не раз ще покажемо Смерть Ворогам.

\normalsize
